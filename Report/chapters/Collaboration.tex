This chapter deals with the collaboration and process in the project. First we introduce some theory on project management and global collaboration. In the next step we report our project work: We introduce the project team, the initial situation and present the chosen project management approach, methods and tools. In addition we evaluate the collaboration tools we used in the project. Finally, we lead to the issues we had to face during the project. These include progress issues, issues with the used project management techniques and collaboration issues. We analyze the collaboration issues based on the previous introduced theory. Furthermore we reflect the process and collaboration and explain our learning outcomes.

%---

\section{Theory}

\subsection{Project management}

There are two major approaches how you can perform project management: The traditional approach and the agile approach.

The traditional approach is ... \todo{TODO}

An agile project management approach is ... \todo{TODO}

A popular agile project management framework is Scrum. \todo{TODO}

\subsection{Global collaboration} \todo{TODO}

	\begin{itemize}
		\item Trust
		\item Individualism
		\item Power distance
	\end{itemize}

%---

\section{Introduction of the Project team}

The project team consists of two groups of students. One group is from the Strathmore University located in Nairobi (Kenya). The other group is from the IT University (ITU) in Copenhagen (Denmark). The project team agreed on to name the two groups "Team Kenya" for the student group from Strathmore University and "Team ITU" for the student group from ITU. This helped to address each group in meeting reports, emails and conversations.

In the beginning Team Kenya consists of three members, which are all in the Masters programme "Telecommunication and Innovation" of the Strathmore University. The members are all from Kenya. The official languages in Kenya are English and Swahili. The members of Team Kenya met each other the first time on the 1.10.2013 (--> Global Meeting Report 1.10.2013 \todo{Add reference}). One member had to leave the group in the last third of the project due to workload of other projects and obligations (--> Global Meeting Report 26.11.2013 \todo{Add reference}).

Team ITU started with five members, which are all in the Masters programme "Software Development and Technology (Software Engineering)" of the ITU. The members are from three different countries: Lithuania, Germany and Denmark. Although there are minor differences between the nationalities, which could have an influence on the team work, we will not go into this, because it is out of scope for this report. The communication language within Team ITU is English, which is not the mother tongue of any of the members. The members of Team ITU met each other the first time on the 27.8.2013. One member left the project after two weeks, because he changed to another project and project team.

Each member of the project team created a member profile to introduce themselves, which are attached in the --> Appendix xx\todo{Add reference}.

Team ITU and the advisor did not get the contact details of the members of Team Kenya until the 17.9.2013. So the collaboration between the teams could not start before this date.

%---

\section{Initial situation}

The students of Team ITU have to complete the project under the course "Global Software Development Project", which is mandatory for their masters programme. The requirements and deadlines for the project are given in the course base from ITU (--> Link to the course base \todo{Add reference}) and by the advisor of the project. The course is rated with 15 ECTS points, which corresponds to approximately 20 hours per week per student. The students of Team ITU have to hand in this report as a mandatory requirement.

For the students from Team Kenya the project is not included in a mandatory course and is voluntarily for the department, which is responsible for the masters programme "Telecommunication and Innovation". There are no mandatory requirements or deadlines, which the Kenyan students have to achieve, except that they have to create documentation for their advisor to prove the progress of the project. Team Kenya agreed on to go with the deadlines from Team ITU (--> Global Meeting Report 24.11.2013 \todo{Add reference}).

As the course for the Team ITU started in late August and the project team and topic was already known, Team ITU already started with the project work before Team Kenya. Team ITU and their advisor did not know when to get the contact details from the student group in Kenya, so the advisor recommended already initializing the project and doing some research and thoughts on the project.

At the beginning Team ITU had received a different project topic by the advisor. The topic of the first project was "Vector Shooter". In this project a program had to be developed, in which the calculation of a vector based on an image has to be made. The image should contain a person, who uses his hands to demonstrate a vector by taking a position like he uses a bow and arrow. The requirement was to use webcams, to capture the image, and to use RaspberryPis to detect the hands of the person and to calculate the vector. Furthermore, an Android application should be included in the programme. Team ITU put some thoughts into the project and spent time on defining the programme, which they wanted to develop. The general idea was to build a game. Some of the thoughts, which were written down, are attached in the --> Appendix \todo{Add reference}.

After three weeks the advisor had to change the topic of the project due to the collaboration with the Strathmore University. Only one student from the Strathmore University was interested on the project "Vector Shooter". Team ITU could have spent time and effort in convincing the other students from Strathmore University to do the project "Vector Shooter". In consideration of the given deadline of the course and on recommendation of the advisor, Team ITU decided to not take this option.

%---

\section{Project Management}

\subsection{Approach}

As the course for the team ITU started in late August and the project team was already known, Team ITU already started with a project management technique, which was already established when Team Kenya got into the project.

Team ITU chose an agile project management approach, because it is more flexible, which was important due to the fact, that Team ITU did not knew the student group from Strathmore University neither their skills nor their requirements on this project. Moreover an agile project management approach is known for being more suitable for software development projects (*TODO: reference*\todo{Find reference}).

Team ITU planned to merge Team Kenya into their project management approach as they did not come up with a different approach. Team ITU considered introducing Scrum to the project. However, due to the inexperience and non-knowledge about this framework in both teams, this idea has been dropped. For Scrum it is necessary that every project team member knows the concept. Thus, each team member would need to learn Scrum, which in turn would have meant losses for the implementation.

%---

\subsection{Methods}

	\begin{itemize}
		\item Milestone plan
		\item Internal weekly meeting (Jour-Fixe)
		\item Global weekly meeting (Jour-Fixe)
		\item Meeting reports
		\item Resource planning
		\item Skill-/Preference-Sheet
		\item Motivation-Sheet
		\item “Out of Office” calendar
		\item Time recording
		\item Kick-Off meeting
	\end{itemize}

%---

\subsection{Collaboration Tools}

	\begin{itemize}
		\item Communication
			\subitem Skype (excluding Google Hangout)
			\subitem Email
		\item File Sharing
			\subitem Google Drive (excluding Skydrive)
			\subitem Github
		\item Time recording
			\subitem Toggl (excluding Excel)
	\end{itemize}

%---

\section{Project Organization}

	\begin{itemize}
		\item Timeline
		\item Project tasks / Roles
	\end{itemize}

%---

\section{Project Issues}

\subsection{Collaboration Issues}
%The issues and how you responded to them

	\begin{itemize}

		\item Illusion of the project work and project team
		\item Failure to comply with the assignments
		\item Communication
		\item Lack of skills
		\item Other exams/hand-ins
		\item Differing requirements
		\item Attendence of meetings
		\item Equipment

	\end{itemize}

%---

\section{Hypothetical Scenarios}
%Relevant hypothetical scenarios

	\begin{itemize}
		\item Assignments
		\item Communication
		\item Requirements
		\item Organisation by the universities (Requirments, clarification, )
	\end{itemize}
