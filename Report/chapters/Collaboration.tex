This chapter deals with the collaboration and process in the project. We introduce the project team, the initial situation and present the chosen project management concept. The procedures and tools, which were used in the project, are declared and evaluated. We also list the difficulties, which occurred during the project and within the global collaboration. Furthermore we reflect the process and collaboration and explain our learning outcomes.


\section{Introduction of the Project team}
The project team consists of two groups of students. One group is from the Strathmore University located in Nairobi (Kenya). The other group is from the IT University (ITU) in Copenhagen (Denmark). The project team agreed on to name the two groups "Team Kenya" for the student group from Strathmore University and "Team ITU" for the student group from ITU. This helped to address each group in meeting reports, emails and conversations.

In the beginning Team Kenya consists of three members, which are all in the Masters programme "Telecommunication and Innovation" of the Strathmore University. The members are all from Kenya. The official languages in Kenya are English and Swahili. The members of Team Kenya met each other the first time on the 1.10.2013 (--> Global Meeting Report 1.10.2013). One member had to leave the project in the last third of the project due to workload of other projects and obligations (--> Global Meeting Report 26.11.2013).

Team ITU started with four members, which are all in the Masters programme "Software Development and Technology (Software Engineering)" of the ITU. The members are from three different countries: Lithuania, Germany and Denmark. Although there are minor differences between the nationalities, which could have an influence on the team work, we will not go into this, because it is out of scope for this report. The communication language within Team ITU is English, which is not the mother tongue of any of the members. The members of Team ITU met each other the first time on the 27.8.2013. One member left the project after two weeks, because he changed to another project and project team.
Each member of the project team created a member profile to introduce themselves, which are attached in the --> Appendix xx.


\section{Initial situation}
The students of Team ITU have to complete the project under the course "Global Software Development Project", which is mandatory for their masters programme. The requirements and deadlines for the project are given in the course base from ITU (--> Link to the course base) and by the advisor of the project. The course is rated with 15 ECTS points, which corresponds to approximately 20 hours per week per student. The students of Team ITU have to hand in this report as a mandatory requirement.

%The motivation of Team Kenya is different? Mandatory course are volunteer project?

There are no mandatory requirements or deadlines, which the Kenyan students have to achieve, except that they have to create a documentation for their advisor to prove the progress of the project. Team Kenya agreed on to go with the deadlines from Team ITU (--> Global Meeting Report 24.11.2013).


\section{Projectmanagement Method}

	\begin{itemize}
		\item Classic approach
		\item SCRUM method
		\item Our method (Structure)
			\item Weekly meetings internal, global, with advisor
			\item Time recording
			\item Splitting up assignmnts
			\item Plans
	\end{itemize}


\section{Collaboration tools}

	\begin{itemize}
		\item Communication
		\item File sharing tools
		\item Time recording
	\end{itemize}


\section{Project Team and Organisation}

	\begin{itemize}
		\item Project team members
		\item Skills
		\item Preferences
		\item Roles
		\item Splitting of the technical parts
	\end{itemize}



\section{Timeline}
%The timeline and how the collaboration evolved along this

	\begin{itemize}
		\item Overview of the phases, milestones, deadlines, other exams/hand-ins
		\item Ressource planning
		\item Time-Recording (Toggl)
	\end{itemize}



\section{Collaboration Issues}
%The issues and how you responded to them

	\begin{itemize}

		\item Illusion of the project work and project team
		\item Failure to comply with the assignments
		\item Communication
		\item Lack of skills
		\item Other exams/hand-ins
		\item Differing requirements
		\item Attendence of meetings
		\item Equipment

	\end{itemize}



\section{Hypothetical Scenarios}
%Relevant hypothetical scenarios

	\begin{itemize}
		\item Assignments
		\item Communication
		\item Requirements
		\item Organisation by the universities (Requirments, clarification, )
	\end{itemize}
