There are numerous factors that must be considered when designing a surveillance system.\begin{enumerate}
\item \textit{\textbf{Position of the camera.}} Camera's type, position and angle may differ a lot, depending on the requirements of the surveillance application. In our scenario it was determined that the most optimal position of the camera would be the ceiling, making the camera point down to the monitored area. This would provide us with an increased field of view, and would make detection of multiple people walking side by side easier. However, due to resource limitations and difficulties we would face trying to position the camera in such way, decision was made to simply place the camera on a higher floor and point it down to the monitored area in approximately 50-60\si{\degree} angle.

\item \textit{\textbf{Object Extraction.}} As we have already touched a bit in Related Work section, background subtraction plays an important role in detecting people and tracking their movement. There are multiple ways in dealing with background subtraction, but we have mainly covered two - simple background subtraction - where we use a static background image to detect difference in frames - and running average approach - where we gradually create an approximate background image to extract moving objects from the frame. The former approach can work rather poorly and inaccurately in dynamic environments, whereas the latter one deals with the same challenges in a better way, thus have been chosen for our design. This is further discussed in greater detail in Section~\ref{sec:object_extraction}.

\item \textit{\textbf{Object Detection.}} To build up a complete and reliable surveillance system, one must be able to identify whether the detected object is a person, an animal or simply tree leaves moving because of wind. To deal with this, one can possibly apply face detection methods or object's shape recognition techniques. Since these are rather complicating areas in image processing, and our test scenarios mainly dealt with environments where the background is more stable, we did not consider object identification and simply interpreted all changes in the monitored area as human movement. Our object detection approach is explained in more detail in Section~\ref{object_detection}.

\item \textit{\textbf{Object Differentiation.}} Another important part of surveillance is being able to differentiation between multiple people crossing the monitored area. To deal with this challenge, we looked into two approaches, namely histogram approach - where we calculate individual histogram for every person and try to match it next time we detect people in the frame - and last coordinate approach - where we, instead of calculating person's histogram, calculate person's last coordinates in the monitored area. The main problem these both approaches share is that they are not very precise, however, the latter approach was easier to implement, thus was chosen for our design. Further comparison and discussion on object differentiation is provided in Section~\ref{object_differentiation}.
\end{enumerate}