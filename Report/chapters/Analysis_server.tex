As there is several devices which have to communicate between each other, we have considered two different solutions:

\begin{enumerate}
\item Client-Server, and
\item Peer-To-Peer (P2P)
\end{enumerate}

To decide which solution suited this project the most, each possible solution has been considered from the following perspectives.

\begin{description}
\item[Storage.] When considering storage we consider how easy it will be to store and retrieve data.
\item[Reliability.] When considering reliability we consider how reliable the communication will be between the cameras and the android devices.
\item[Scalability.] When considering scalability we consider how the system will perform with different amount of devices connected.
\item[Cost.] When considering cost we will consider how much it cost mostly in development time due to complexity and to a lesser degree prices of such implementation.
\end{description}

\subsection{Client-Server}
The Client-Server model consists of two specialized applications: A client and a server. The server application will provide some shared resources. The client can initiate contact to a server and use the provided resources.

The considered advantages and disadvantages of the Client-Server model are as follows:

\begin{description}
\item[Storage.] Storing data at a central point reduces data flow on the network and availability to old data for all devices. It also makes it easy to backup data.

\item[Reliability.] Performs well if there is a heavy load on the network. A server is build to handle a lot of connections and each peer will only have to manage its own connection to the server. If a server is heavily burdened an additional server can be set up and share the load. Also it is only the server that has to be available for all devices which reduces potential problems with different network configurations. A drawback of using a server is that it introduces single point of failure risking that if the server fails the whole system will be down.

\item[Scalability.] Performs well with both few and many connections. If the server is overloaded an additional server can be set up increasing the capacity. This model does not overload the network with redundant data as the devices only have to send the data once.

\item[Cost.] As server and client is more specialized it is easier to develop and maintain each separate application. Also with this split the server can be updated without third party users have to update.
\end{description}

\subsection{P2P}
The P2P model is a decentralized and distributed network model. It consists of several devices called "peers". Each peer can acts as both server and client which share or store some resources.

The considered advantages and disadvantages of the P2P model are as follows:

\begin{description}
\item[Storage.] Distributed data storage is possible, but it is very complex and overall space required will be very high due to redundancy.

\item[Reliability.] P2P is very reliable as there is no single point of failure. If a peer drops out the data can be provided from another peer. If a Raspberry PI disconnects its web-camera will of course not be available, but the system will continue working. A potential problem is different firewall configurations which might result in the peer not being available for other peers at all.

\item[Scalability.] P2P does not scale very good in a system like this as the web-camera peers will either have to send to a lot of peers or the peers will get the data delayed. If a peer requests some old data it can be distributed from another peer reducing traffic to the raspberry Pis.

\item[Cost.] Only one application will be developed and maintained, however, this single application will be very complex.
\end{description}

\subsection{Comparison}
\begin{description}
\item[Storage.] Central storage has a lot of advantages compared to distributed. It consumes less overall space and it creates less network traffic. As most data in this system is used for live tracking a distributed data model will lead to a huge amount of redundant data which will only be used once. For these reasons central storage is preferable.

\item[Reliability.] Both solutions can be considered reliable, however a server rarely brakes down and with a lot of firewalls all over the place the chance that a peer will not be available for other peers will probably be higher than a server going down.

\item[Scalability.] In a system like this a Client-Server approach will scale a lot better than the P2P since it will require a lot of clients to force a second server and the P2P model will result in either overload of the Raspberry PIs or delayed data deliveries.

\item[Cost.] P2P will be incredible complex to develop compared to the Client-Server. Also as the system will be developed by two teams located at different places, a split will make it easier to split up task, without both teams having to update every time the other one finds a bug.

\end{description}

All in all the Client-Server model is much more suited for this type of system than the P2P.