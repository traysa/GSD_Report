\section{Context}
\label{sec:Context}

%given whatever a 1st semester MSc student from ITU knows
%What the audience needs to know in order to start reading this report.
%To bring the reader up to a level that allows you to tell your story.
%Course setup
%Aim for 1 page.

Smart use of energy resources is an ongoing topic these days. The reduction of expenses is mostly the biggest driving factor for companies. But also the debate around climate change brings new legislation to reduce the waste of energy resources, whose production is damaging to the environment and future generations. The IT University of Copenhagen (ITU) has an interest in producing an occupancy model for commercial buildings, like the ITU building, to detect where energy resources are needed and where they can be saved. Energy resources are needed for e.g. lighting and heat-regulating systems, which are relevant for occupants in a commercial building. With the detected occupancy data the ITU can predict occupancy and develop concepts for a smart use of energy resources in commercial buildings.

The Strathmore University in Kenya has also an interest in building up an occupancy model, but mainly for surveillance reasons. Surveillance can be used for several purposes like traffic monitoring, public safety and facilities surveillance. An IT-based surveillance system can automatically analyze the scene without the use of human resources. By analyzing the scene the detection of occupancy is a major part. Moreover a real-time prediction model on top of the occupancy data can be used to prevent criminal activity by triggering alarms or other surveillance systems.

Currently there is no existing infrastructure to build up an occupancy model in the Strathmore University or the ITU building. Both universities want a solution for an occupancy analyzer based on Raspberry Pi computers due to the minimal consumption of computational and monetary resources. Furthermore, Starthmore University requests for an Android application, which represents a live-feed of the occupants in a monitored room.

A group of students from both universities have to collaborate to come up with a solution for an occupancy analyzer, which can satisfy the needs of both university interests. Ideally a product should be developed, which can be adapted to fit the needs of one or the other university.
Furthermore, a collaboration project is mandatory for the student group from ITU, in which they have to face the challenges of global collaboration, navigate compromises and come up with a solution.

This report contains the product result, design of the product, details of the project work and the learning outcomes, which were achieved in the project with the globally distributed team from the perspective of the ITU students. The project team consists of international students located in Nairobi, Kenya (East African Time) and Copenhagen, Denmark (Central European Time).


\section{Problem}
\label{sub:problem}
%given the context
%What is the specific problem you are looking at? At this point this won't be clear to you and that is okay. Spend some time thinking about what the system you are building could be used for and write down something that makes sense (notice that this is the reverse approach: what do we want to do -> what does this solve).This makes you think about the problem and gives us something to discuss. Later on we will revise it. 
%Describe the problem you are facing.
%Give a clear definition of the problem you are trying to solve. This is what the success of the product is measured against.
%The context
%Aim for ½-1 page.

The content of this project is to build an occupancy analyzer, which detects people in a room or corridor and predicts their movement. The occupancy analyzer has to be based on Raspberry Pi computers, which come with computational restrictions, and web cameras. A solution for the right architecture and programming languages has to be found, which can deal with those limited resources.

Due to the usage of web cameras, a visual detection of people has to be made. Analyzing images by detecting people - which are moving objects and not part of the room - is one major challenge to face. Visual conditions of the image can change as a result of daylight. For example, dynamic lighting and moving shadows should not influence the detection of people. 
To best capture occupants and the requirement to represent the occupants on an Android application, the position of web cameras becomes an important factor.
The differentiation between multiple people, as well as between people and the setting of the room, is important for the quality of the detection. Only if reliable data about occupancy exists, can the data be used to construct a reliable prediction model usable for future concepts and projects.

The question of how the collected data can be used, has to be considered. Building prediction models, which relies on historical data and real-time data, is another requirement, which the project dealt with. Decisions on what kind of prediction for a meaningful application - like the one mentioned in paragraph 2 in Section~\ref{sec:Context} - and how the data will be stored and processed have to be made.

Besides the design and implementation of an occupancy analyzer, another task is the collaboration of students from two different universities. Cultural differences, difference in time, spatial distance and locally related influences have to be overcome. Different perspectives have to be combined and compromises have to be made.

\section{Related Work}
\label{sub:related}
%given the poblem
%What have other people done that is relevant to your problem and what can we learn from them. Start by writing a list of paragraphs. Each should describe one related work, where it did well in respect to your specific problem and where it falls short of solving this. 
%Looking back, what have others done that is relevant to the problem?
%To show that you have done your homework and surveyed existing solutions.
%The problem
%The Internet; in particular: ACM http://dl.acm.org; IEEE http://ieeexplore.ieee.org
%What is the core of each contribution?
%How does this apply to your specific problem?
%What can you build on and what is missing?
%Aim for 1-2 pages.

%NREL IPOS Project:
%http://techportal.eere.energy.gov/technology.do/techID=986
%http://www.nrel.gov/news/features/feature_detail.cfm/feature_id=2210
%http://e3tnw.org/Documents/IPOS%20Presentation.pdf
%http://www.ibpsa.us/presentations/2013.06_IBO_Boulder/Thursday/Low-Cost%20Sensing/Low-Cost%20Sensing_L_GENTILE_POLESE.pdf
%http://m.eetasia.com/ART_8800686058_480500_NT_509225b4.HTM


%http://ieeexplore.ieee.org/xpl/login.jsp?reload=true&tp=&arnumber=5606357&url=http%3A%2F%2Fieeexplore.ieee.org%2Fiel5%2F30%2F5606236%2F05606357.pdf%3Farnumber%3D5606357

%The concept of an occupancy analyzer, which uses visual recognition of occupants, is already handled in several papers and projects. \\
%Some guy investigated something about some image analysis and or occupancy analysis. \\

Background subtraction is an extremely important concept in surveillance applications, since this is the first step in detecting people and tracking their movement in monitored area. Y. Benezeth, P.-M. Jodoin, B. Emile, H. Laurent \& C. Rosenberger~\cite{simple_background_subtraction}, in their work \emph{Comparative Study of Background Subtraction Algorithms}, mention a basic motion detection approach by using a simple background subtraction technique. In this technique, the background image is taken when there is a total absence of motion in the monitored area. This background image is then subtracted from every newly taken image of the area in order to detect motion. This is possible since after subtraction the resulting image will either be almost totally black - meaning no motion occurred - or the image will have some resulting bright contours of detected differences.

We intend to apply this simple technique in the beginning of the project for the initial experiments and studies of object detection and tracking. We expect that these experiments and studies will provide us with some knowledge on how basic object detection works and how we can apply it for our specific problem.

Aditi Jog \& Shirish Halbe~\cite{jog_halbe}, in their paper \emph{Multiple Object Tracking using CAMshift Algorithm in Open CV}, present a better way of building object detection and tracking application than simply using the basic background subtraction approach. For their implementation they build upon an existing algorithm called CAMshift~\footnote{Continuously Adaptive Mean Shift.}, which is capable of tracking multiple moving objects in video streams. In this algorithm, an approximate background image of the monitored area is gradually built up by smoothing or blurring out the motion areas. This is done by using several recent images of the monitored area, and performing arithmetic averaging on them. One can then simply use this built up background image for detecting changes in the monitored area.

This approach applies to the first part of our problem, namely detecting and tracking people in the monitored area. We can build upon this approach as the implementation of it is quite simple yet flexible and adaptive. Moreover, it eliminates the need to worry about fluctuations in lighting and other unpredictable changes in the background that may corrupt the interpretation of the monitored area.

Gellert \& Vintan~\cite{gellert}, in their work \emph{Person Movement Prediction Using Hidden Markov Models}, presents the use of hidden Markov models to anticipate the next movement of a given person. It is used inside an office building where it is possible to predict the next room given any room based on the history of rooms visited by a person moving within the building. The configuration of the model is optimised by evaluating movement sequences of real persons within an office building.

The concept of calculating the probability, and thereby making a prediction, of the next room given any room can be related to our scenario where we predict the next occupied section of the image given an occupant in any section. This information is used in the process of predicting where an occupant will exit the room.

Ashbrook \& Starner~\cite{ashbrook} demonstrated how locations of significance can be incorporated into a predictive model in the paper \emph{Using GPS to Learn Significant Locations and Predict Movement Across Multiple Users}. They cluster location data from a GPS over an extended period of time and infer meaningful locations. These locations are then incorporated into Markov models where it is used to predict a person's future location given any current location. The idea of using heavily visited areas, that is \emph{locations of significance}, to perform predictions is transferred to our scenario, where each area of the image holds a value reflecting the amount of activity within the area. This value plays an important role in the calculations of probabilities and thereby the final predictions.

\section{Approach}

%given the problem and the related work
%How do you intend to solve the problem? 
%To give a summary of how you did things.
%How do you propose to solve the problem? Or at least, how do you propose to take a step towards a solution for the problem?
%Aim for 1 page.
Given the problems listed in section \ref{sub:problem} and the related work listed in section \ref{sub:related}, we propose a system consisting of three webcams, three Raspberry Pis, a single server and an android application. Each webcam is placed in a seperate room recording occupant activity. The webcam is connected to a Raspberry Pi performing analysis on the images received from the cameras. The analysis is made to detect occupants within the image and the information is sent and stored on the server. The android application can request information about the locations of each occupant and predictions about a single occupant. A prediction is essentially a guess about where an occupant will exit a room based on historical data. At any point during the presence of an occupant a prediction can be requested. \\
Due to the limited computation capabilities of the Raspberry Pis we have to perform benchmarks to find the appropriate programming language and framework for image analysis. For occupant detection, we gradually build up an approximate background image of the monitored area by smoothing and blurring out the areas with activity. The background image is compared to a given image where the difference displays where exactly activity has taken place. \\
We choose to follow a client-server approach where all communication from and to the devices and database goes through the server. The prediction logic is stored on the server as well and will be based on the approach laid out by Gellert \& Vintan\cite{gellert}. Instead of transitions between office rooms we split the image into sections and calculate probabilities of transitioning to other image sections. Our prediction approach is derived from the hidden Markov Models where we use historical data the past amount of activity in each image section in the calculations. \\
The design and implementation of the android application is assigned to the Kenyan team with whom we will come to mutual agreements about the data sent from the server and what prediction information can be requested. These decisions will be agreed upon during weekly meetings and regular email correspondences.

\section{Report Structure}
%This is one easy paragraph that you write the last week.
%How is the report structured?
%Give a quick hint of the layout of the report
%For each top-level section write a sentence describing the section. Make sure those sentences are linked.
Section \ref{analysis} analyses the problems presented in section \ref{sub:problem} and presents and discusses various solutions. Based on the discussions, section \ref{design} presents the final system design. Any implementation specific details about the system that are deemed not immediately obvious will be covered in section \ref{implementation}. Section \ref{evaluation} evaluates the results from tests of the final system. The whole project process including management, issues and collaboration with the Kenyans is covered in section \ref{collaboration}. Finally, the system and project process will be concluded in section \ref{conclusion}.