\section{Context}
\label{sec:Context}

%given whatever a 1st semester MSc student from ITU knows
%What the audience needs to know in order to start reading this report.
%To bring the reader up to a level that allows you to tell your story.
%Course setup
%Aim for 1 page.

Smart use of energy resources is an ongoing topic these days. The reduction of expenses is mostly the biggest driving factor for companies. But also the debate around climate change brings new legislation to reduce the waste of energy resources, whose production is damaging to the environment and future generations. The IT University of Copenhagen (ITU) has an interest in producing an occupancy model for commercial buildings, like the ITU building, to detect where energy ressources are needed and where they can be saved. Energy resources are needed for e.g. lighting and heat-regulating systems, which are relevant for occupants in a commercial building. With the detected occupancy data the ITU can predict occupancy and develop concepts for a smart use of energy resources in commercial buildings.

The Strathmore University in Kenya has also an interest in building up an occupancy model, but mainly for surveillance reasons. Surveillance can be used for several purposes like traffic monitoring, public safety and facilities surveillance. An IT-based surveillance system can automatically analyse the scene without the use of human ressources. By analysing the scene the detection of occupancy is a major part. Moreover a real-time prediction model on top of the occupancy data can be used to prevent criminal activity by triggering alarms or other surveillance systems.

Currently there is no existing infrastructure to build up an occupancy model in the Strathmore University or the ITU building. Both universities want a solution for an occupancy analyzer based on Raspberry Pis due to the minimal consumption of computational and monetary ressources. Furthermore, Starthmore University requests for an Android application, which represents a live-feed of the occupants in a monitored room.

A group of students from both universities have to collaborate to come up with a solution for an occupancy analyzer, which can satisfy the needs of both university interests. Ideally a product should be developed, which can be adapted to fit the needs of one or the other university.
Furthermore, a collaboration project is mandatory for the student group from ITU, in which they have to face the challenges of global collaboration, navigate compromises and come up with a solution.

This report contains the product result, design of the product, details of the project work and the learning outcomes, which were achieved in the project with the globally distributed team from the perspective of the ITU students. The project team consists of international students located in Nairobi, Kenya (East African Time) and Copenhagen, Denmark (Central European Time).


\section{Problem}

%given the context
%What is the specific problem you are looking at? At this point this won't be clear to you and that is okay. Spend some time thinking about what the system you are building could be used for and write down something that makes sense (notice that this is the reverse approach: what do we want to do -> what does this solve).This makes you think about the problem and gives us something to discuss. Later on we will revise it. 
%Describe the problem you are facing.
%Give a clear definition of the problem you are trying to solve. This is what the success of the product is measured against.
%The context
%Aim for ½-1 page.

The content of this project is to build an occupancy analyzer, which detects people in a room or corridor and predicts their movement. The occupancy analyzer has to be based on Raspberry Pis, which comes with computational restrictions, and webcams. A solution for the right architecture and programming languages has to be found, which can deal with those limited ressources.

Due to the usage of webcams, a visual detection of people has to be made. Analyzing images by detecting people - which are moving objects and not part of the room - is one major challenge to face. Visual conditions of the image can change as a result of daylight. For example dynamic lighting and moving shadows should not influence the detection of people. 
To best capture occupants and the requirement to represent the occupants on an Android application, the webcam positions becomes important.
The differentiation between multiple people, as well as between people and the setting of the room, is important for the quality of the detection. Only if reliable data about occupancy exists, can the data be used to construct a reliable prediction model usable for future concepts and projects.

The question of how the collected data can be used, has to be considered. Building prediction models, which relies on historical data and real-time data, is another requirement, which the project dealed with. Decisions on what kind of prediction for a meaningful application - like the one mentioned in paragraph 2 in section \ref{sec:Context} - and how the data will be stored and processed have to be made.

Besides the design and implementation of an occupnacy anaylzer, another task is the collaboration of students from two different located universities. Cultural differences, difference in time, spatial distance and locally related influences have to be overcome. Different perspectives have to be combined and compromises have to be made.


\section{Related Work}

%given the poblem
%What have other people done that is relevant to your problem and what can we learn from them. Start by writing a list of paragraphs. Each should describe one related work, where it did well in respect to your specific problem and where it falls short of solving this. 
%Looking back, what have others done that is relevant to the problem?
%To show that you have done your homework and surveyed existing solutions.
%The problem
%The Internet; in particular: ACM http://dl.acm.org; IEEE http://ieeexplore.ieee.org
%What is the core of each contribution?
%How does this apply to your specific problem?
%What can you build on and what is missing?
%Aim for 1-2 pages.

The concept of an occupancy analyzer, which uses visual recognition of occupants, is already handled in several papers and projects. \\
Some guy investigated something about some image analysis and or occupancy analysis. \\

Gellert \& Vintan~\cite{gellert}, in their work \emph{Person Movement Prediction Using Hidden Markov Models}, presents the use of hidden Markov models to anticipate the next movement of a given person. It is used inside an office building where it is possible to predict the next room given any room based on the history of rooms visited by a person moving within the building. The configuration of the model is optimised by evaluating movement sequences of real persons within an office building. \\
The concept of calculating the probability, and thereby making a prediction, of the next room given any room can be related to our scenario where we predict the next occupied section of the image given an occupant in any section. This information is used in the process of predicting where an occupant will exit the room. \\
Ashbrook \& Starner~\cite{ashbrook} demonstrated how locations of significance can be incorporated into a predictive model in the paper \emph{Using GPS to Learn Significant Locations and Predict Movement Across Multiple Users}. They cluster location data from a GPS over an extended period of time and infer meaningful locations. These locations are then incorporated into Markov models where it is used to predict a person's future location given any current location. The idea of using heavily visited areas, that is \emph{locations of significance}, to perform predictions is transferred to our scenario, where each area of the image holds a value reflecting the amount of activity within the area. This value plays an important role in the calculations of probabilities and thereby the final predictions.

NREL IPOS Project:
%http://techportal.eere.energy.gov/technology.do/techID=986
%http://www.nrel.gov/news/features/feature_detail.cfm/feature_id=2210
%http://e3tnw.org/Documents/IPOS%20Presentation.pdf
%http://www.ibpsa.us/presentations/2013.06_IBO_Boulder/Thursday/Low-Cost%20Sensing/Low-Cost%20Sensing_L_GENTILE_POLESE.pdf
%http://m.eetasia.com/ART_8800686058_480500_NT_509225b4.HTM


%http://ieeexplore.ieee.org/xpl/login.jsp?reload=true&tp=&arnumber=5606357&url=http%3A%2F%2Fieeexplore.ieee.org%2Fiel5%2F30%2F5606236%2F05606357.pdf%3Farnumber%3D5606357



\section{Approach}

%given the problem and the related work
%How do you intend to solve the problem? 
%To give a summary of how you did things.
%How do you propose to solve the problem? Or at least, how do you propose to take a step towards a solution for the problem?
%Aim for 1 page.



\section{Report Structure}

%This is one easy paragraph that you write the last week.
%How is the report structured?
%Give a quick hint of the layout of the report
%For each top-level section write a sentence describing the section. Make sure those sentences are linked.