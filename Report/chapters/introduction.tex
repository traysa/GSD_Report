\section{Context}

%given whatever a 1st semester MSc student from ITU knows
%What the audience needs to know in order to start reading this report.
%To bring the reader up to a level that allows you to tell your story.
%Course setup
%Aim for 1 page.

Smart use of energy ressources is an ongoing topic these days. The reduction of expenses is mostly the biggest impulse for companies. But also the debate around climate change brings a new legislation to reduce the waste of energy resources, whose production is damaging to the environment and future generations. The IT University of Copenhagen (ITU) has an interest in producing an occupancy model for commercial buildings, like the ITU building, to detect where energy ressources are needed and where it can be saved. Energy ressources are needed for e.g. lighting and heat-regulating systems, which are relevant for occupants in a commercial building. With the detected occupancy data the ITU can predict occupancy and develop concepts for a smart use of energy ressources in commercial buildings.

The Strathmore university in Kenya has also an interest in building up an occupancy model, but mainly for surveillance reasons. Surveillance can be used for several purposes like traffic monitoring, public safety and facilities surveillance. An IT-based surveillance system can automatically analyse the scene without the use of human ressources. By analysing the scene the detection of occupancy is a major part. Moreover a real-time prediction model on the occupancy data can be used for preventing criminal behaviour by triggering alarm or other surveillance systems.

Currently there is no existing infrastructure to build up an occupancy model in the Strathmore university or the ITU building. Both universities want a solution for an occupancy analyzer based on Raspberry PIs due to the minimal consumption of computational and monetary ressources. 

A group of students from both universities have to collaborate to come up with a solution for an occupancy analyzer, which can satisfy the needs of both university interests. Ideally a product should have been developed, which can be adapted to fit one or the other university needs.
Furthermore a collaboration project is mandatory for the student group from ITU, in which they have to face the challenges of global collaboration, navigate compromises and come up with a solution.

This report contains the product result, details of the project work and the learning outcomes, which were achieved in the project with the globally distributed team. The project team consists of international students located in Nairobi, Kenya (East African Time) and Copenhagen, Denmark (Central European Time).


\section{Problem}

%given the context
%What is the specific problem you are looking at? At this point this won't be clear to you and that is okay. Spend some time thinking about what the system you are building could be used for and %write down something that makes sense (notice that this is the reverse approach: what do we want to do -> what does this solve).This makes you think about the problem and gives us %something to discuss. Later on we will revise it. 
%Describe the problem you are facing.
%Give a clear definition of the problem you are trying to solve. This is what the success of the product is measured against.
%The context
%Aim for ½-1 page.



\section{Related Work}

%given the poblem
%What have other people done that is relevant to your problem and what can we learn from them. Start by writing a list of paragraphs. Each should describe one related work, where it did well in respect to your specific problem and where it falls short of solving this. 
%Looking back, what have others done that is relevant to the problem?
%To show that you have done your homework and surveyed existing solutions.
%The problem
%The Internet; in particular: ACM http://dl.acm.org; IEEE http://ieeexplore.ieee.org
%What is the core of each contribution?
%How does this apply to your specific problem?
%What can you build on and what is missing?
%Aim for 1-2 pages.



\section{Approach}

%given the problem and the related work
%How do you intend to solve the problem? 
%To give a summary of how you did things.
%How do you propose to solve the problem? Or at least, how do you propose to take a step towards a solution for the problem?
%Aim for 1 page.



\section{Report Structure}

%This is one easy paragraph that you write the last week.
%How is the report structured?
%Give a quick hint of the layout of the report
%For each top-level section write a sentence describing the section. Make sure those sentences are linked.